\documentclass[12pt]{article}

\usepackage[margin=1in]{geometry}
\usepackage{mathptmx}
\usepackage[colorlinks, citecolor=black, linkcolor=black, %
filecolor=black, urlcolor=blue]{hyperref}
\usepackage{natbib}
\usepackage{doi}
\usepackage{amsmath}

\title{TMR: A parallel mesh generation and adaptation tool}
\author{Graeme J. Kennedy}
\date{}

\begin{document}

\maketitle

\section{Introduction}

TMR is a tool for creating meshes for 2D, mapped 2D, and 3D geometries which can then be refined using a forest of quadtrees in 2D or octrees in 3D.
The quadrilateral or hexahedral meshes created by TMR can be either second- or third-order.
TMR is designed to facilitate the creation of a hierarchical series of meshes for multigrid methods.
The refinement algorithms within TMR can be used with solution or design-based refinement or adjoint-based refinement techniques.
The interface to the underlying geometry kernel within TMR is an abstract layer that provides both geometry and topology information available from CAD.
This abstract interface has been implemented using OpenCascade so that STEP files can be loaded.

The meshing capabilities in TMR are build around two steps:
%
\begin{enumerate}
\item The creation of moderate size quadrilateral and hexahedral meshes in serial using a combination of the blossom-quad algorithm for 2D meshes and swept-mesh methods for 3D meshes.
\item Adaptive quadtreee and octree meshing and partitioning in parallel based on the input connectivity from the moderate size 2D and 3D meshes.
\end{enumerate}
%
These capabilities can be used to analyze and optimize large-scale problems efficiently on high-performance computing resources. 

This document describes the algorithms and methods used in TMR and its interface.
It is divided between a high-level description of the algorithms and a detailed description of the interface in TMR.

\section{Algorithms}

The following sections describe the algorithms that are implemented with TMR at a high-level.
More detailed descriptions can be found within the comments in the code.

\subsection{Serial meshing capabilities}

The 2D and mapped 2D meshing capabilities in TMR are built around the blossom-quad algorithm~\citep{Remacle:2012:blossom-quad}.
This algorithm generally produces good quality quadrilateral meshes with some restrictions on the input geometry and the mesh size.
The maximum mesh size is approximately 1/2 the minimum geometric feature size or edge length, or 1/4 the minimum hole size.
This requirement arises because there is a minimum of 2 elements per edge or 4 elements around a hole.
Sharp corners with small acute angles limit local mesh quality and lead to the introduction of local kite elements. 

Blossom-quad first builds a triangular surface mesh which is then converted to a quadrilateral mesh through an optimal weighted matching using the blossom algorithm.
TMR uses a generalization of Rebay's algorithm for efficient unstructured triangular mesh generation~\citep{Rebay:1993:unstructured-triangle}. 
This algorithm generates a triangular surface mesh using a frontal method that updates a Delaunay triangularization based on the Bowyer--Watson algorithm~\citep{Shewchuk:2012:Delaunay-notes}.
Once the triangular mesh is generated, it is smoothed using a Laplacian technique.
The recombination of the triangular mesh into a quad mesh is then performed using the weighted matching algorithm. 
The weights between adjacent triangles are computed based on a combined quad quality, which is a function of the maximum interior angle in the quadrilateral.
The weights are a nonlinear function of this quality metric, where angles close to $180^{\circ}$ are most heavily penalized.
Adjacent triangular elements along the boundary are also linked with imaginary quads with a large weight.

The matching algorithm selects which quads to combine in order to create the mesh with the best overall quality based on the weights. 
The quad mesh is then post-processed to remove poor quality quadrilateral elements with specific topologies.
Finally, the resulting quad mesh is smoothed using a quad-specific algorithm by~\citet{Giuliani:1982:quad-smoothing}. 
The smoothing algorithm minimizes a combination of the squeeze and distortion in the mesh. 

\subsection{Mapped Quadtree and octree meshing}

Quadtree and octree meshing techniques provide an intermediate step between fully structured meshes and unstructured meshes. 
There are several ways that an quadtree or octree can be stored.
The most natural way is to store the data in a tree structure where the leaves of the tree represent elements in the mesh.
However, this type of data structure method requires extra overhead and memory needed to traverse through the depth of the tree.

In TMR, the leaves of the quadtree or octree are stored in an array data structure directly, without referencing the nodes within the tree hierarchy. 
This approach is based on the work of~\citet{BursteddeWilcoxGhattas11} and \citet{IsaacBursteddeWilcoxEtAl15}.
Within this data structure, the leaves represent either elements or nodes within the mesh.
To mesh realistic geometries, the method employs a semi-structured approach in which a global unstructured quadrilateral or hexahedral super-mesh defines the connections between quadtree or octrees.
Each quadtree or octree is then parametrically mapped to the geometry using an abstract geometry layer described below.

Each quadrant or octant within a local quadtree is uniquely identified based on its local $x$-$y$-$z$ coordinates and its refinement level. 
The coordinates correspond to the lower left hand corner of the quadrant or octant and the level is used to determine the length of the side of an element. 
This length can be used to determine the local coordinates of neighboring elements.
The tree can be traversed by creating parent or child quadrants. 

To enable local changes in the mesh refinement, adjacent quadrants may have different levels, producing elements with different side-lengths.
Dependent or hanging nodes must be added when the mesh refinement level changes between adjacent quadrants. 
To ensure finite-element compatibility between adjacent elements, dependent nodes must be constrained along an edge.  
To ensure compatibility, TMR imposes that the relative difference in levels between two adjacent quadrants cannot exceed one. 
This is referred to as 2:1 balancing. 
After a refinement step, the global mesh must be balanced so that both intra- and inter-quadtree quadrants are 2:1 balanced. 
Inter-quadtree operations are performed by mapping the quadrants from their locally-aligned coordinate system to the coordinate system of an adjacent quadtree along common shared edges or corners.

In addition to the element mesh, the semi-structured method must also create and uniquely order the nodes. 
A local Morton encoding on each tree to facilitate these operations. 
The order of the quadrilaterals within the global quadtree mesh is based on the unstructured global mesh. 
The ownership of shared corners, edges, and faces is determined in advance to avoid duplicating node numbers. 
During the node ordering process, hanging nodes are labeled and their corresponding independent neighbors are determined so that they can be eliminated using compatibility constraints during finite-element analysis.

\subsection{Interface to CAD}

The interface to CAD geometry is provided through an intermediate interface in TMR. 
An implementation of this interface is provided for OpenCascade which enables models to be loaded directly from OpenCascade or intermediate tools. 
For instance, OpenCascade can load geometry that has been exported in a STEP format. 

TMR has internal definitions for both geometry and topological entities.
The geometric entities include nodes, curves, and surfaces, while the topological entities include vertex, edge, edge loop, face, and volume objects.
The topological entities describe the logical relationships between their underlying geometric representations.
Each vertex is associated with a node, each edge with a curve, and each face with surface.
Each edge contains a first and second vertex denoting the start and end point of the underlying curve.
An edge loop contains a series of edges that are connected together to form a closed loop on a surface.
Each face is bounded by a series of edge loops that define the extent of the face and any holes or cutouts that form the surface.
The volume object contains a series of faces that bound the volume to create a watertight surface.

The orientation of geometry objects within the model are of critical importance.
Incorrect orientation information can produce a model with an incompatible topology that does not reflect the underlying geometry.
Each edge has a natural orientation defined by its first and second vertices.
However, within an edge loop, the orientation of a particular edge may be reversed relative to its natural orientation. 
Therefore, the edge loop object stores both the list of edges that form the loop and their orientations relative to the natural edge orientation.

Faces in TMR are defined in their natural orientation.
This means that the parametric areas computed are positive such that the parametric system is right-handed.
The orientation of an edge loop on a face is always defined such that the material lies to the left of an edge within an edge loop when walking the loop in its positive orientation (taking into account the relative orientation in the edge loop object). 
The natural orientation may vary from the orientation used within the CAD package and so TMR stores a flag to indicate whether the orientation of the surface is flipped.
The surfaces that create a volume have an orientation, defined with an outward normal direction.
The volume object contains a list of surfaces and their orientations relative to the natural surface orientation stored in TMR that indicates whether the surface is outward facing.

The meshes within TMR are generated and stored on a component-level basis.
Meshing proceeds through the hierarchy of topological entities from vertices, edges, faces to volumes.
Each component stores its own portion of the mesh using its natural orientation.
However, when the mesh is extracted, the orientation is converted to the CAD-based orientation by flipping the orientation when it does not align. 

\section{TMR Interface and Methods}

There are two interfaces to TMR: the C++ interface and the Python-level interface. 
TMR is implemented in C++, so the interface through C++ contains all publicly accessible class member functions.
The Python-level interface wraps the most important classes and functions, but does not provide an interface to all lower-level operations.

There are three primary types of classes within TMR:
\begin{enumerate}
\item Geometry and topology-level classes which interface with the underlying geometry kernell
\item Serial mesh-level classes which are used to create moderate-size quadrilateral and hexahedral meshes based on the input geometry; and
\item Quadtree and Octree-level classes which are used to create, modify, refine and extract information from the quad/octrees.
\end{enumerate}
%
The methods and data stored in each of these types of classes will be described below.

Almost all classes in TMR inherit from \texttt{TMREntity}. 
Those that do not are low-level data structures that are used within higher-level operations.
The \texttt{TMREntity} base class defines a reference counting scheme for all objects allocated on the heap. 
Once an object is allocated, you must call \texttt{obj->incref()} and if it goes out of scope you must call \texttt{obj->decref()}.
These calls are handled automatically by the Python interface.

\subsection{Geometry-level classes}

The primary geometry-level classes used within TMR consist of entity-level classes and container-level classes. The abstract interface classes to the underlying geometry consist of:
\begin{itemize}
\item \texttt{TMRVertex}: Evaluate a node location
\item \texttt{TMREdge}: Evaluate a node location on a curve with a single parametric argument $t$. Provides access to the first and second vertices that begin/end the edge. Given a parametric location $t$ and a face, return the $(u,v)$ location on the face.
\item \texttt{TMRFace}: Evaluate a node location given a parametric location $(u,v)$. Access the edge loops that bound the face.
\end{itemize}

The other objects define collections of geometric entities:
%
\begin{itemize}
\item \texttt{TMREdgeLoop}: Contains an oriented set of edges that enclose a surface. Provides the edge list and their relative orientations in the loop.
\item \texttt{TMRVolume}: Contains an oriented collection of surfaces that enclose a volume. Provides a list of surfaces and their relative orientations in the volume.
\item \texttt{TMRModel}: Contains an ordered collection of vertices, edges, faces, and volumes that define a model geometry.
\end{itemize}

The \texttt{TMRModel} class is often created by a call to \texttt{TMR\_LoadModelFromSTEPFile( const char* )}. 
This function loads the model data from the STEP file and creates the OpenCascade versions of the geometric entities and containers that define the model.
This call attempts to remove any extraneous geometry defined in the STEP file but not contained within the model.
This function creates and initializes the internal topology of the model, which can then be used in subsequent meshing operations.

A direct connection between the geometry and octree or quadtree levels is provided through the \texttt{TMRTopology} class that defines the topology for mapped-quadrilateral and hexahedral geometries. 
This class can be created directly, but it is most common to create this using the mesh-level classes described below.

The class \texttt{TMRPoint}, which does not inherit from \texttt{TMREntity} contains the $x$, $y$, $z$ location of a point.

\subsection{Mesh generation-level classes}

\texttt{TMRMesh} provides the primary interface to the meshing functionality in TMR and is used to create quadrilateral and hexahedral meshes.
The actual meshing operations are performed on the root processor with rank 0, but all connectivity and location information is broadcast to all processors.
The primary functions in \texttt{TMRMesh} are:
\begin{itemize}
\item \texttt{void mesh(double)}: Mesh the model with the provided spacing and default meshing options
\item \texttt{void mesh(TMRMeshOptions, double)}: Mesh the model with the provided spacing and the provided meshing options
\item \texttt{int getMeshPoints(TMRPoint**)}: Retrieve a global array of the mesh locations
\item \texttt{int getMeshConnectivity}: Retrieve the global connectivity from the mesh
\item \texttt{TMRModel *createModelFromMesh()}: Create a geometry model based on the mesh
\item \texttt{void writeToVTK(const char*, int)}: Write the mesh to a VTK file
\item \texttt{void writeToBDF(const char*, int)}: Write the mesh to a BDF file
\end{itemize}

\texttt{TMRMeshOptions} class defines a number of options that modify the meshing algorithm. These include the following:
\begin{itemize}
\item \texttt{int triangularize\_print\_level}: Print level to provide more
  verbosity during the triangularization algorithm.
\item \texttt{int write\_mesh\_quality\_histogram}: Write out a histogram of the mesh quality in the final smoothed quadrilateral mesh.
\item \texttt{TMRFaceMeshType mesh\_type\_default}: Default mesh type either structured or unstructured. In the case that it is set to structured, the algorithm firsts checks if it is possible to use a mapped mesh, and reverts to an unstructured algorithm otherwise.
\item \texttt{int num\_smoothing\_steps}: Number of smoothing steps to apply during both the Laplacian and quad smoothing algorithms
\item \texttt{TriangleSmoothingType tri\_smoothing\_type}: Smooth the triangular mesh using either a Laplacian smoother or a spring-based analogy
\item \texttt{double frontal\_quality\_factor}: Use the mesh quality indicator to determine when to accept new triangles in the frontal algorithm.
\end{itemize}

The mesh itself is created by meshing the vertices, edges, faces, and volumes within the mesh.
These meshes are stored in the following classes
%
\begin{itemize}
\item \texttt{TMREdgeMesh}: Stores the node numbers along an edge
\item \texttt{TMRFaceMesh}: Stores the structured/unstructured surface mesh
\item \texttt{TMRVolumeMesh}: Stores the hexahedral volume mesh
\end{itemize}

TMR only supports swept volume mesh generation.
This requires the specification of additional information to indicate what surfaces should be linked and what direction should be swept.
This information is provided by indicating source and target geometry entities.
When a source/target pair is set, the mesh connectivity is copied from the source to the target. 
A swept volume can only be created if:
\begin{enumerate}
\item The \texttt{TMRVolume} object contains a source/target \texttt{TMRFace} object pair that share the same number of edge loops and each source/target edge pair has the same number of nodes; and
\item All other \texttt{TMRFace} objects are structured with the same number of nodes along the swept direction.
\end{enumerate}
%
To ensure these conditions apply, it is often necessary to set source/target pairs for faces and edges.

The \texttt{TMREdgeMesh} object is created by first computing the number of nodes along its length, and then globally ordering the nodes. The number of nodes is determined based on the following criteria:
\begin{itemize}
\item If the first and second vertices are different, then at least 3 nodes are created along the edge.
\item If the first and second vertices are the same and the edge is not degenerate, then it forms a loop back on itself and at least 5 nodes are created (double counting the first and last node number).
\item If the edge is a target edge, the number of nodes is taken from the source edge.
\item Otherwise, the number of nodes is selected as an odd number that most closely matches the spacing requested along the edge.
\end{itemize}
Note that an odd number of nodes are required so that an even number of elements are created along the edge which is needed to ensure the proper functioning of the blossom-quad algorithm.

\subsection{Quad/octree-level classes}

The quadtree and octree level classes are used to create, refine and manipulate the semi-structured mesh in parallel.
The quadrants or octants are stored in a distributed manner across processors.
The following section describes the functions for the forest of octree operations.
Analogous functionality is defined for the forest of quadtree object as well.
The first step in creating a \texttt{TMROctForest} object is typically creating a \texttt{TMRTopology} object which defines a model with mapped 2D and 3D geometry that consists entirely of non-degenerate quadrilateral surfaces with four non-degenerate edges and hexahedral volumes with 6 enclosing surfaces.
This type of object can be created using the constructor \texttt{TMRTopology( MPI\_Comm, TMRModel* )}.
The \texttt{TMRTopology} object defines additional functionality that is generally required only for lower-level operations.

Once the topology object has been created, the \texttt{TMROctForest} object can be initialized. It has the following functions:
%
\begin{itemize}
\item \texttt{TMROctForest( MPI\_Comm )}: Create the forest of octree object without initializing anything
\item \texttt{void setTopology( TMRTopology* )}: Set the \texttt{TMRTopology} object for the underlying
\item \texttt{void repartition()}: Repartition the mesh across processors. This redistributes the elements so that there are an equal, or nearly equal, number of elements on each processor.
\item \texttt{void createTrees( int refine\_level )}: Create all the octrees in the mesh with the given level of refinement. Be careful, the number of elements in the mesh scales with $8^{\text{level}}$.
\item \texttt{void createRandomTrees()}: Create a set of trees with random levels of refinement. This can be useful for testing purposes
\item \texttt{TMROctForest *duplicate()}: Duplicate a forest by copying connectivity, topology and elements (but not duplicating nodes)
\item \texttt{TMROctForest *coarsen()}: Create a new forest object by coarsening all the elements within the mesh by one level, if possible. Does not create new nodes.
\item \texttt{void refine( const int refinement[] )}: Refine the elements in the node by the specified number of levels. If a negative number is supplied, coarsen the element.
\item \texttt{void balance( int balance\_corner=0 )}: Balance all the elements in the mesh to achieve a 2:1 balance 
\item \texttt{void createNodes()}: Create all the nodes within the mesh
\item \texttt{TMROctantArray* getOctsWithAttribute( const char *attr )}: Get an array of octants with the desired attribute
\item \texttt{int* getNodesWithAttribute( const char *attr, int *len )}: Get an array of the nodes numbers with the desired attribute
\item \texttt{void createMeshConn( int **conn, int *nelems )}: Create and return the distributed mesh connectivity using a global ordering
\item \texttt{void createDepNodeConn()}: Create the dependent (hanging) node connectivity
\item \texttt{int getDepNodeConn( const int **ptr, const int **conn, const double **weights )}: Retrieve the dependent node connectivity from the octree object
\item \texttt{void createInterpolation( TMROctForest *coarse, TACSBVecInterp *interp )}: Create an interpolation object between two octrees that share a common topology.
\end{itemize}

\subsubsection{Typical usage}

The typical usage for a \texttt{TMROctForest} object would consist of the following:
\begin{enumerate}
\item Create the object and call \texttt{setTopology()} to set the super-mesh
\item Create an initial element mesh by calling \texttt{createTrees()}
\item Create a refined mesh by duplicating and refining the octree forest by calling \texttt{duplicate()} followed by \texttt{refine()}. Note that the length of the integer array passed to \texttt{refine} must be equal to the number of elements in the original mesh.
\item Balance the mesh and create nodes by calling \texttt{balance()} then \texttt{createNodes()}
\item Create a sequence of coarser lower-order meshes by repeated calling \texttt{duplicate()} and \texttt{coarsen()}.
\item Create interpolants between meshes for multigrid methods by calling \texttt{createInterpolation()}
\end{enumerate}

\subsection{Python-level}

An interface to TMR through Python has been created using cython.
The Python-level wrapper contains all of the objects described above without the preciding \texttt{TMR}. 
To use the TMR-Python interface requires calls such as
\begin{verbatim}
from mpi4py import MPI
from tmr import TMR

# Set the communicator
comm = MPI.COMM_WORLD

# Load the model from the STEP file
geo = TMR.LoadModel('geometry.stp')

# Get the volumes
vols = geo.getVolumes()

# Get the edges/faces from the geometry
faces = geo.getFaces()
edges = geo.getEdges()

# Set the source/target relationships
faces[4].setSource(vols[0], faces[5])
edges[8].setSource(edges[5])

# Create the geometry
mesh = TMR.Mesh(comm, geo)

# Mesh the part
opts = TMR.MeshOptions()
opts.num_smoothing_steps = 0

# Mesh the geometry with the given target size
htarget = 4.0
mesh.mesh(htarget, opts=opts)
\end{verbatim}

\subsection{Error estimation algorithms}

There are two mesh refinement strategies that are implemented in TMR: the first is based on the strain energy norm, while the second is an adjoint-based refinement technique.
The strain energy based refinement method is designed to reduce the solution error in the natural strain energy norm.
This call takes the form:
%
\begin{verbatim}
double TMR_StrainEnergyErrorEst( TMRQuadForest *forest,
                                 TACSAssembler *tacs,
                                 TMRQuadForest *forest_refined,
                                 TACSAssembler *tacs_refined,
                                 double *error );
\end{verbatim}
%
The forest can be refined based on the requested target error and the solution set in TACS.

The adjoint-based refinement method requires a \texttt{TACSAssembler} object for both the current level of refinement and a uniformly refined version of the model. 
The adjoint-based mesh refinement call takes the form
\begin{verbatim}
double TMR_AdjointErrorEst( TMRQuadForest *forest,
                            TACSAssembler *tacs,
                            TMRQuadForest *forest_refined,
                            TACSAssembler *tacs_refined,
                            TACSBVec *adjoint,
                            double *error,
                            double *adj_corr );
\end{verbatim}
The adjoint vector is supplied in \texttt{adjvec}.

\bibliographystyle{abbrvnat}
\bibliography{refs}

\end{document}
