\documentclass[12pt]{article}

\usepackage[margin=1in]{geometry}
\usepackage{mathptmx}
\usepackage[colorlinks, citecolor=black, linkcolor=black, %
filecolor=black, urlcolor=blue]{hyperref}
\usepackage{natbib}
\usepackage{doi}

\title{TMR: A parallel mesh generation and adaptation tool}
\author{Graeme J. Kennedy}
\date{}

\begin{document}

\maketitle

\section{Introduction}

TMR is a tool for creating meshes for 2D, mapped 2D, and 3D geometries which can then be refined using a forest of quadtrees/octrees.
The quadrilateral or hexahedral meshes created by TMR can be either second- or third-order.
TMR is well-suited to create a hierarchical series of meshes for multigrid methods.
The refinement algorithms within TMR can be used with either on solution or design-based refinement or adjoint-based refinement.
The geometry interface within TMR is designed to be an abstract interface to a general CAD package that provides both geometry and topology information.
This abstract interface has been implemented using OpenCascade so that STEP files can be loaded.

The meshing capabilities in TMR are build around two steps:
%
\begin{enumerate}
  \item The creation of moderate size quadrilateral and hexahedral meshes in serial using a combination of the blossom-quad algorithm for 2D meshes and sweep methods for 3D meshes.
  \item Adaptive quadtreee and octree meshing and partitioning in parallel based on the input connectivity from the moderate size 2D and 3D meshes.
\end{enumerate}
%
These capabilities can be used to analyze and optimize large-scale problems efficiently on high-performance computing resources. 

This document describes the essential internal workings of TMR and its interface.
It is divided between a high-level description of the algorithms and a more-detailed description of the interface in TMR that are

\section{Algorithms}

The following sections describe at a high-level the algorithms implemented with TMR.
More detailed descriptions can be found within the comments in the code.

\subsection{Serial meshing capabilities}

The 2D and mapped 2D meshing capabilities in TMR are built around the blossom-quad algorithm~\citep{Remacle:2012:blossom-quad}.
This algorithm generally produces good quality quadrilateral meshes with some restrictions on the input geometry and the mesh size.
The maximum mesh size is approximately 1/2 the minimum geometric feature size or edge length, or 1/4 the minimum hole size.
This requirement arises because there is a minimum of 2 elements per edge or 4 elements around a hole.
Sharp corners with small acute angles limit local mesh quality and lead to the introduction of local kite elements. 

Blossom-quad first builds a triangular surface mesh and then converts the mesh to a quadrilateral mesh through an optimal weighted matching using the blossom algorithm.
TMR uses a generalization of Rebay's algorithm for efficient unstructured triangular mesh generation~\citep{Rebay:1993:unstructured-triangle}. This algorithm generates a triangular surface mesh using a frontal algorithm and keeps a Delaunay triangularization based on the Bowyer--Watson algorithm~\citep{Shewchuk:2012:Delaunay-notes}.
Once the triangular mesh is generated, it is smoothed using a Laplacian smoothing technique.
The recombination of the triangular mesh into a quad mesh is then performed using the weighted matching algorithm. 
The weights between adjacent triangles are computed based on a combined quad quality, which is a function of the maximum interior angle in the quadrilateral.
The weights are a nonlinear function of this quality metric, where angles close to $180^{\circ}$ are most heavily penalized.
Adjacent triangular elements along the boundary are also linked with imaginary quads with a large weight.

The matching algorithm selects which quads to combine in order to create the mesh with the best overall quality based on the weights. 
The quad mesh is then post-processed to remove poor quality quadrilateral elements with specific topologies.
Finally, the resulting quad mesh is smoothed using a quad-specific algorithm by~\citet{Giuliani:1982:quad-smoothing}. The smoothing algorithm minimizes a combination of the squeeze and distortion in a mesh. 

\subsection{Mapped Quadtree and octree meshing}

Quadtree and octree meshing techniques provide an intermediate step between fully structured meshes and unstructured meshes. 
There are several ways that an quadtree or octree can be stored.
The most natural way is to store the data in a tree structure where the leaves of the tree represent elements in the mesh.
However, this type of data structure method requires extra overhead and memory compared to scan through the depth of the tree.

In TMR, the leaves of the quadtree or octree are stored in an array data structure directly, without referencing the nodes within the tree hierarchy. 
This approach is based on the work of~\citet{BursteddeWilcoxGhattas11} and 
\citet{IsaacBursteddeWilcoxEtAl15}.
Within this data structure the leaves represent either elements or nodes within the mesh.
To mesh realistic geometries, the method employs a semi-structured approach in which a global unstructured quadrilateral or hexahedral super-mesh defines the connections between quadtree or octrees.
Each quadtree or octree is then parametrically mapped to the geometry using an abstract geometry layer described below.

Each quadrant or octant within a local quadtree is uniquely identified based on its local $x$-$y$-$z$ coordinates and its refinement level. 
The coordinates correspond to the lower left hand corner of the quadrant or octant and the level is used to determine the length of the side of an element. 
This length can be used to determine the local coordinates of neighboring elements.
The tree can be traversed by creating parent or child quadrants. 

To enable local changes in the mesh refinement, adjacent quadrants may have different levels, producing elements with different side-lengths.
Dependent or hanging nodes must be added when the mesh refinement level changes between adjacent
quadrants. 
To ensure finite-element compatibility between adjacent elements, dependent nodes must be constrained along an edge.  
To ensure compatibility, TMR imposes that the relative difference in levels between two adjacent quadrants cannot exceed one. 
This is referred to as 2:1 balancing. 
After a refinement step, we balance the full global mesh so that both intra- and inter-quadtree quadrants are 2:1 balanced. 
Inter-quadtree operations are performed by mapping the quadrants from their locally-aligned coordinate system to the coordinate system of an adjacent quadtree along common shared edges or corners.

In addition to the element mesh, the semi-structure method must also create and uniquely order the nodes. 
A local Morton encoding on each tree to facilitate these operations. 
The order of the quadrilaterals within the global quadtree mesh is based on the
unstructured global mesh. 
The ownership of shared corners, edges, and faces is determined in advance to avoid duplicating node numbers. 
During the node ordering process, hanging nodes are labeled and their corresponding independent neighbors are determined so that they can be eliminated using compatibility constraints during finite-element analysis.

\subsection{Interface to CAD}

The interface to CAD geometry is provided through an intermediate interface in TMR. 
An implementation of this interface is provided for OpenCascade which enables models to be loaded directly from OpenCascade or intermediate tools. 
For instance, OpenCascade can load geometry from other CAD tools that has been exported in a STEP format. 

TMR has internal definitions for both geometry and topological entities.
The geometric entities include a node, curve, and surface objects, while the topological entities include a vertex, edge, edge loop, face, and volume objects.
The topological entities describe the logical relationships between their underlying physical geometric representations.
Each vertex is associated with a node, each edge with a curve, and each face with surface.
Each edge contains a first and second vertex denoting the start and end point of the underlying curve.
An edge loop contains a series of edges that are connected together to form a closed loop on a surface.
Each face is bounded by a series of edge loops that define the extent of the face and any holes or cutouts that form the surface.
The volume object contains a series of faces that bound the volume to create a watertight surface.

The orientation of geometry objects within the model are of critical importance.
Incorrect orientation information can produce a model with an incompatible topology that does not reflect the underlying geometry.
Each edge has a natural orientation defined by its first and second vertices.
However, within an edge loop, the orientation of a particular edge may be reversed relative to its natural orientation. 
Therefore, the edge loop object stores both the list of edges that form the loop and their orientations relative to the natural edge orientation.

Faces in TMR are defined in their natural orientation.
This means that the parametric areas computed are positive such that the parametric system is right-handed.
The orientation of an edge loop on a face is always defined such that the material lies to the left of an edge within an edge loop when walking the loop in its positive orientation (taking into account the relative orientation in the edge loop object). 
The natural orientation may vary from the orientation used within the CAD package and so TMR stores a flag to indicate whether the orientation of the surface is flipped.

Finally, the surfaces that create a volume have an orientation, defined with an outward normal direction.
The volume contains a list of surfaces and their orientations relative to the natural surface orientation stored in TMR that indicates whether the surface is outward facing.

The meshes within TMR are generated and stored on a component-level basis.
Meshing proceeds through the hierarchy of topological entities from vertices, edges, faces to volumes.
Each component stores its own portion of the mesh using its natural orientation.
However, when the mesh is extracted, the orientation is converted to the CAD-based orientation by flipping the orientation when it does not align. 

\section{TMR Interface and Methods}


\bibliographystyle{abbrvnat}
\bibliography{refs}

\end{document}